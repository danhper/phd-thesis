\chapter{Conclusion}

\label{ch:conclusions}

In this chapter, we summarise chapter-by-chapter the achievements of our thesis.

\point{Execution layer}
At the execution layer, we contributed to improving the security of the Ethereum Virtual Machine (EVM), and in particular, the gas metering mechanism.

Our approach involved creating an instrumented version of the EVM that allowed us to replay and analyze the execution of smart contracts.
By examining several months' worth of transactions, we identified several discrepancies in the metering model, including significant inconsistencies in the pricing of instructions.
Furthermore, we discovered that there was a weak correlation between execution cost and the resources utilized, such as CPU and memory.

As a result, we introduced a new type of DoS attack, known as the ``Resource Exhaustion Attack'' that leveraged these weaknesses to generate low-throughput contracts.
To demonstrate the vulnerability of major Ethereum client implementations, we designed a genetic algorithm that generated contracts with a throughput on average \Slowdown times slower than typical contracts.
Our findings indicated that if these clients were running on commodity hardware, they would be unable to remain synchronized with the network when subjected to this attack.

\point{Transactional layer}
Our focus at the transactional layer was on blockchains with higher scalability and their transactional throughput.
To conduct our analysis, we examined transaction data for three high-scalability blockchains: EOSIO, Tezos, and XRP Ledger (XRPL) over a period of seven months.

Our findings revealed that only a small portion of transactions was utilized for value transfer purposes.
Specifically, 96\% of transactions on EOSIO resulted from airdrops of a token without any current value.
In the case of Tezos, 76\% of throughput was utilized for maintaining consensus, while over 94\% of transactions on XRPL had no economic value. We also identified a persisting airdrop on EOSIO that qualified as a DoS attack and detected a two-month-long spam attack on XRPL.

We also explored how the different designs of the three blockchains had impacted user behaviour.
Through this analysis, we gained insights into utilization patterns of transactional throughput and how blockchain designs could influence user behaviour.

\point{Application layer}
At the application layer, we focused on the DeFi ecosystem, and first formalised the concepts of technical and economic security.

We then analysed the technical security of smart contracts deployed on Ethereum.
We analysed 20 million transactions interacting with contracts flagged vulnerable by program analysis tools and found that at most \ExploitedEther ETH (\ExploitedEtherUSD USD) of the \EtherClaimedVulnerable ETH (\EtherClaimedVulnerableUSD USD) potentially at risk was exploited, which represents a mere 0.27\%.
We investigated these contracts in more depth and found that most were not exploited in practice because of the lack of feasibility of the exploit or because of the lack of economic incentive to do so.

Finally, we studied the economic security of DeFi lending protocols and found that users often have very risky positions, with variations as small as 3\% in an asset's price being able to turn over 10 million USD worth of collateral liquidatable.
We also found that the efficiency of the liquidations has increased over time and that at the time of the analysis, over 70\% of the liquidations were instant.
Lastly, we also found that depegging events of stablecoin have caused very large amounts of liquidations because of the over-confidence in their stability.
