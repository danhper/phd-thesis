\chapter{Background Theory}
\label{ch:background}

In this chapter, we will provide background about blockchains, smart contracts, and their applications that will be useful for understanding the rest of the thesis.
More specific background information, such as some more details about the Ethereum platform, will be provided in the relevant chapters.

\section{Blockchain fundamentals}
In its essence, a blockchain is an append-only, decentralized database that is replicated across multiple computer nodes. 
Most blockchain systems record activities in the form of ``transactions''. 
A transaction typically contains information about its sender, its receiver, as well as the action taken, such as the transfer of an asset. 
Newly created transactions are broadcast across the network where they get validated by the participants. 
Valid transactions are grouped into data structures called \textit{blocks}, which are appended to the blockchain by referencing the most recent block.
Blocks are immutable, and state changes in the blockchain require new blocks to be produced.

Network latency and asynchrony inherent in the distributed nature of blockchains lead to a number of challenges. 
In particular, a blockchain must be able to reach consensus about the current state when the majority of participating nodes behave honestly. 
In order to resolve the disagreement, a consensus protocol prescribing a set of rules is applied as part of the validation process. 

To ensure consistency, there needs to be:

\begin{enumerate}\itemsep=-1pt
\item a set of rules to validate transactions \label{req1};
\item a set of rules to validate blocks \label{req2};
\item a mechanism to determine which chain of blocks represents the current state \label{req3}
\end{enumerate}

%
In the following subsections, we will provide definitions and explanations about transactions, blocks and consensus mechanisms, and then describe how the above requirements are fulfilled in most existing blockchain systems.

% Requirement (\ref{req1}) is usually straightforward to implement, each participant simply verifies that the set of rules holds for each transaction.
% Most blockchains also have a simple solution for Requirement (\ref{req3}): they treat the longest existing chain of blocks as the valid one.
% The biggest challenge, and diversity of solutions arise when it comes to solving Requirement (\ref{req2}).
% Bitcoin provides the first practical solution to this problem, namely, Proof-of-Work~(PoW).
% Participating nodes who propose new blocks compete to solve a computationally expensive puzzle, and blocks are deemed valid if Requirement (i) they include a solution to this puzzle and (ii) all transactions they contain are valid.

\subsection{Transactions and blocks}
\point{Transactions} The smallest unit of work in most blockchain systems, including Bitcoin and Ethereum, is a transaction.
The exact content of a transaction varies between different systems. It typically contains information about the sender, the receiver, the amount of the asset being transferred, and information on the fees to be paid for the transaction to be processed.
In the case of Ethereum, transactions can also contain arbitrary data, which can be used to invoke smart contracts, which we will cover in more depth further in this chapter.
As part of the set of rules to fulfil Requirement \ref{req1} above, transactions are signed by the sender, and the signature is used to verify that the transaction is valid.
The systems also verify that the sender has enough funds to cover the transaction, including its associated fees, and that the transaction is not a duplicate of a previously processed transaction.
Finally, in the case of a smart contract interaction, the execution of the smart contract must also be successful.

\point{Blocks}
A block is a collection of transactions that are grouped together and appended to the blockchain.
Blocks typically contain some extra metadata, such as a hash of the previous block, effectively linking them together, a timestamp, and information about the transactions included in the block.
Each block is usually limited to a maximum number of transactions, and a block is considered valid if all transactions it contains are valid.
However, this is not the only requirement for a block to be deemed valid since this requirement alone would be prone to double-spending.
To fulfil Requirement \ref{req2} above, some consensus algorithm needs to be used.
We will discuss the most common consensus algorithms in the next subsection.

\subsection{Consensus}
\point{Proof-of-Work}
The Proof-of-Work (PoW) consensus, introduced by Bitcoin requires the participant to solve a computationally expensive puzzle to create a new block. Although PoW can maintain consistency well, it is by nature very time- and energy-consuming, which limits its throughput.
To preserve security while maintaining a sufficient degree of decentralization, scalability is often sacrificed~\cite{Xie2019}.
Indeed, the rate of block creation for both Bitcoin is relatively slow---on average 10 minutes per block, respectively---and the only way to increase the throughput is to increase the size of a single block, allowing for more transactions per block.

\point{Proof-of-Stake} Along with PoW, Proof-of-Stake (PoW), to which Ethereum has recently transitioned, is another consensus mechanism that solves the same issue without requiring a large amount of computational power.
PoS requires its participants, called block proposers and validators, to stake a certain amount of their assets to be eligible to create a new block.
The probability of a block proposer being selected to create a new block is proportional to the total amount of assets they have staked.
The block proposer is then required to create a new block and broadcast it to the network.
The validators then verify the block and vote on whether it is valid.
If the block is invalid, the block proposer is penalised, by taking some of the assets it has staked, and the block is discarded.
If a majority of the validators agree that the block is valid, it is appended to the blockchain.

\point{Deciding on the valid chain}
These consensus mechanisms alone are not enough to fulfil Requirement \ref{req3}.
In both PoW and PoS, there could be two competing versions of the blockchain containing a different chain of blocks.
For PoW, this could for example happen if two miners have found a solution to the puzzle at the same time.
The most common solution to this problem is to treat the longest chain of blocks as the valid one.
In the case of PoW, the longest chain is not exactly the one that contains the most blocks but the one that contains the most work, i.e., the one that requires the most computational effort to produce.
In the case of PoS, the issue is more complex since it does not require any computational effort to produce a block.
This means that in theory, someone could produce a large number of blocks to create the longest chain.
To prevent this, multiple mechanisms exist but the one used by Ethereum is to force blocks to be produced within a fixed interval of time and to use some blocks as checkpoints that cannot be reverted.


\subsection{Incentives and fees}
For both PoW and PoS to work, the miners, or block proposers for PoS, need to be incentivised to participate in the consensus process.
This is typically done by rewarding them with a certain amount of the asset when they produce a new block, and also giving them the transaction fees of the transactions included in the block.
For Bitcoin, the incentive mechanism is straightforward: the miner receives a fixed block reward, of which the amount halves every four years, and the sum of all the transaction fees of transactions in the block.
Ethereum used to work in the same way but has since changed to a more complex incentive mechanism.
One of the main differences with the Bitcoin model is that the block reward is not fixed but instead varies with the total amount of assets staked in the system.
Another notable difference is that part of the fees are not distributed to the block proposer but instead are burned, i.e., they are destroyed.

\section{Smart contracts}
The Ethereum~\cite{buterin2014} platform was the first to allow its users to run ``smart contracts'' on its distributed infrastructure.
Smart contracts are programs that define a set of rules for the governing of associated funds, typically written in a Turing-complete programming language, usually Solidity~\cite{Dannen:2017:IES:3103305} in the case of Ethereum.
Solidity is similar to JavaScript, yet some notable differences are that it is strongly typed and has built-in constructs to interact with the Ethereum platform.
Programs written in Solidity are compiled into low-level untyped bytecode to be executed on the Ethereum platform by the Ethereum Virtual Machine (EVM)~\cite{wood2014ethereum}.
It is important to note that it is also possible to write EVM contracts without using Solidity.

To execute a smart contract, a sender has to send a transaction to the contract and pay a fee which is derived from the contract's computational cost, measured in units of~\emph{gas}. Each executed instruction consumes an agreed-upon amount of gas~\cite{wood2014ethereum}. Consumed gas is credited to the miner of the block containing the transaction, while any unused gas is refunded to the sender. In order to avoid system failure stemming from never-terminating programs, transactions specify a gas limit for contract execution. An out-of-gas exception is thrown once this limit has been reached.

Smart contracts themselves have the capability to \emph{call} another account present on the Ethereum blockchain. This functionality is overloaded, as it is used both to call a function in another contract and to send Ether (ETH), the underlying currency in Ethereum, to an account. A particularity of how this works in Ethereum is that calls from within a contract, also called \emph{internal transactions}, do not create new transactions and are therefore not directly recorded on-chain. This means that looking at transactions without executing them does not provide enough information to follow the flow of Ether.


\section{Applications}

\point{DAOs} 
One of the earliest applications of distributed ledger technology was the creation of Decentralized Autonomous Organizations (DAOs).
A DAO is a decentralized organization that uses a blockchain to facilitate the management of its funds and decision-making process.
It often uses at least a multi-signature wallet to manage its funds, which means that several members of the organisation are required to approve any transfer of funds.
DAOs also often have more complex rules, such as delays to be able to perform some actions or recurrent payments, that can be enforced by smart contracts.
The blockchain executes the rules and decision-making process in a transparent manner, resulting in a system requiring less trust than its centralised counterparties.
DAOs can serve a variety of functions, such as managing decentralized funds, investing in projects, creating decentralized social networks, or facilitating community governance.
\point{DeFi}
Decentralized Finance (DeFi) is a peer-to-peer financial system powered by a blockchain that is designed to operate without the need for traditional financial intermediaries such as banks, brokerages, or exchanges.
In a DeFi system, financial transactions and services are facilitated by smart contracts that automatically execute transactions and enforce the terms of agreements.
Examples of DeFi applications include decentralized exchanges (DEXs), where users can trade cryptocurrencies without the need for a centralized exchange, lending platforms that enable borrowers to access loans without going through a traditional bank, and stablecoins that are designed to maintain a stable value.
We will discuss DeFi in depth in Chapter~\ref{ch:application-security}.
\point{Others} There are many other applications of blockchain technology, such as the tokenisation of art, often through non-fungible tokens (NFT), decentralized identity management, decentralized storage, and decentralized social networks.
Although these are important applications, they are not the focus of this thesis and will not be discussed in detail.

\section{Examples of Attacks on Blockchain Systems}

Over the years, blockchain systems have been the target of many attacks.
These attacks have happened at different levels of the system, from the consensus protocol to the smart contracts.
In this section, we provide an example attack for each of the layers that we cover in this thesis: execution layer, transactional layer, and application layer.
Furthermore, for the application layer, we provide an example of a technical attack and an example of an economic attack.
Formal definitions for these terms are provided in Chapter~\ref{ch:application-security}.

\point{Shanghai attack}
In September 2016, a DoS attack was performed on the Ethereum network.
The attack was performed during a major Ethereum conference taking place in Shanghai, hence the name of the attack.
The gist of the attack was an attacker flooding Ethereum with transactions that were cheap to execute but would take the network participants a long time to process.
This resulted in the network taking much longer than normal to produce new blocks and forced Ethereum to revisit their pricing mechanism for transactions.
We give more in-depth details about this attack in Chapter~\ref{ch:vm-security}.

\point{Solana DDoS}
While the Shanghai attack is a good example of a DoS attack that relies on abusing the resource pricing mechanism of a transaction, other blockchains have also been the target of DDoS attacks despite the resource pricing mechanism working as intended.
This happens most often in blockchains that try to have very low transaction fees, such as Solana~\cite{solana}, where a single transaction costs less than a cent.
In September 2021, an attacker performed a DDoS attack on Solana by sending over 400,000 transactions per second, which is vastly over the network's capacity.
Because of the extremely low transaction price, the cost of the attack was very moderate for the attacker.
The Solana network ended up having to rollback the network to be able to recover.
We discuss similar types of attacks and discuss in-depth the trade-offs of low transaction fees in Chapter~\ref{ch:transactional-security}.

\point{TheDAO hack}
TheDAO exploit~\cite{Securities2017} is one of the most infamous exploits that occurred in an Ethereum smart contract.
Attackers exploited a re-entrancy vulnerability~\cite{Atzei2017} of the contract which allowed for the draining of the contract's funds.
The attacker contract could call the function to withdraw funds in a re-entrant manner before its balance on TheDAO was reduced, making it indeed possible to freely drain funds.
A total of over 3.5 million Ether were drained.
Given the severity of the attack, the Ethereum community finally agreed on hard forking.
We discuss this type of vulnerability in depth in Chapter~\ref{ch:application-security}.

\point{Aave bad debt}
