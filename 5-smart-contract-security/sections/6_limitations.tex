\section{Limitations}
\label{sec:limitations} 
In this section, we present the different limitations of our system, and describe how we try to mitigate them.

\point{Soundness vs Completeness} As for most tools such as this one, we are faced with the trade-off of soundness against completeness. Whenever possible we choose soundness over completeness --- three out of six of our analyses are sound. When we cannot have a sound analysis, we are careful to only leave out cases which are unlikely to generate many false negatives. In other word, we try as much as possible to reduce the number of false negatives, even if this means increasing the number of false positives. Indeed, the main goal of our system is to provide us an upper-bound of the amount of potentially exploited Ether, which make false negatives undesirable. Furthermore, we manually check the high-value contracts flagged as exploited, which assures us that even if false-positives were flagged, they will not have an important influence on the final results. As an example of this trade-off, for re-entrancy we flag any contract which was called using a re-entrant call and lost funds in the process. However, in some cases, it could be an expected behavior and the funds could have been transferred to an address belonging to the same entity.

\point{Dataset} We only run our tool on the contracts included in our dataset, which means that we might be missing some exploits which actually occurred. Indeed, we did not have any contract affected by the Parity wallet bug nor had we the contract affected by TheDAO hack in the dataset. However, one of the main goal of this paper is to quantify what fraction of vulnerabilities discovered by analysis tools is exploited in practice and our current approach allows us to do exactly this.

\point{Other types of attacks} Our tool and analysis does not cover every existing attack to smart contracts. There are, for example, attacks targeting ERC-20 tokens~\cite{8802438}, or yet some other types of DoS attacks, such as wallet griefing~\cite{Grech2018}.
Furthermore, some detected ``exploits'' could be the results of Honeypots~\cite{236240} but our tool does not cover such cases.
Although it would be interesting to also cover such cases, we had to make a decision about the scope of the tool. Therefore, we focus on the vulnerabilities which have been the most covered in the literature, which we hypothesise is representative of how common the vulnerabilities are.
