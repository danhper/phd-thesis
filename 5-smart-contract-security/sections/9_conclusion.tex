\section{Conclusion}
\label{sec:conclusion}

In this paper, we surveyed the~\VulnerableContracts vulnerable contracts reported by~\PapersAnalyzed recent academic projects. We proposed a Datalog-based formulation for performing analysis over EVM execution traces and used it to analyze a total of more than~\empirical{20 million} transactions executed by these contracts. We found that at most~\NumExploitedContracts out of~\VulnerableContracts contracts have been subject to exploits but that at most~\ExploitedEther ETH (\ExploitedEtherUSD USD), or only~\PercentExploitedEther of the \EtherClaimedVulnerable ETH (\EtherClaimedVulnerableUSD USD) potentially at risk, was exploited.
Finally, we found that a majority of Ether is held by only a small number of contracts and that the vulnerabilities reported on these are either false positives or not exploitable in practice, thus providing a reasonable explanation for our results.

% Our results suggest that the impact of vulnerable smart contracts on the Ethereum blockchain had been exaggerated. We hypothesize that the main reasons for the significant gap between vulnerable and exploited are: many contracts flagged vulnerable are not in practice exploitable and most of the high-value contracts, which would be the most interesting to the attackers, fall into this category.
