% \begin{abstract}
% Metering is an approach developed to assign cost to smart contract execution in blockchain systems such as Ethereum. This paper presents a detailed investigation of the metering approach based on \emph{gas} taken by the Ethereum blockchain. We discover a number of discrepancies in the metering model such as  significant inconsistencies in the pricing of the instructions. We further demonstrate that there is very little correlation between the gas and resources such as CPU and memory. We find that the main reason for this is that the gas price is dominated by the amount of \emph{storage} that is used.

% Based on the observations above, we present a new type of DoS attack we call~\emph{Resource Exhaustion Attack}, which uses these imperfections to generate low-throughput contracts. Using this method, we show that we are able to generate contracts with a throughput on average~\Slowdown times slower than typical contracts. We then show that all major Ethereum clients implementations are vulnerable. Such malicious contracts can be used to prevent nodes with lower hardware capacity from participating in the network, thereby artificially reducing the level of decentralization the network can deliver. Our attack has been reported to the Ethereum Foundation and awarded a bug bounty reward of~5,000~USD.
% \end{abstract}

\begin{abstract}
  Blockchain systems, such as Ethereum, use an approach called ``metering'' to assign a cost to smart contract execution, an approach which is designed to incentivise miners to operate the network and protect it against DoS attacks. In the past, the imperfections of Ethereum metering allowed several DoS attacks which were countered through modification of the metering mechanism.

  This paper presents a new DoS attack on Ethereum which systematically exploits its metering mechanism. We first replay and analyse several months of transactions, during which we discover a number of discrepancies in the metering model, such as significant inconsistencies in the pricing of the instructions. We further demonstrate that there is very little correlation between the execution cost and the utilised resources, such as CPU and memory. Based on these observations, we present a new type of DoS attack we call~\emph{Resource Exhaustion Attack}, which uses these imperfections to generate low-throughput contracts. To do this, we design a genetic algorithm that generates contracts with a throughput on average~\Slowdown times slower than typical contracts. We then show that all major Ethereum client implementations are vulnerable and, if running on commodity hardware, would be unable to stay in sync with the network when under attack. We argue that such an attack could be financially attractive not only for Ethereum competitors and speculators, but also for Ethereum miners. Finally, we discuss short-term and potential long-term fixes against such attacks. Our attack has been responsibly disclosed to the Ethereum Foundation and awarded a bug bounty reward of~5,000~USD.
\end{abstract}
