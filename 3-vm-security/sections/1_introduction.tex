\section{Introduction}
\label{sec:3:introduction}

Some blockchain systems support code execution, allowing arbitrary programs to take advantage of decentralised trust.
Ethereum and its virtual machine, the Ethereum Virtual Machine (EVM), is probably the most widely used blockchain adopting this approach.
However, allowing arbitrary programs from non-trusted users introduces many new challenges.
One of these challenges is to prevent users from running code which could negatively impact the performance of the system.
To tackle this challenge, Ethereum introduced the notion of ``gas'', which is a unit used to measure the execution cost of a program, referred to as a ``smart contract'' in this context.
Gas-based metering is used to price the execution of smart contracts and must ensure that the throughput of the blockchain, in terms of gas per second, remains stable.
Metering is therefore critical to keep the Ethereum blockchain safe against Denial of Service (DoS) attacks involving slow-running contracts.
However, assigning costs to different instructions is a highly non-trivial task, and the costs originally assigned in the Ethereum yellow paper~\cite{wood2014ethereum}, which were designed to maintain a throughput of 1 gas/$\mu$s, had many inconsistencies.
As a consequence, several DoS attacks have been conducted on Ethereum~\cite{transaction-spam-attack,suicide-attack}, and the gas cost has also been reviewed several times~\cite{erc150,eip-1884} to increase the cost of the under-priced instructions.

To the best of our knowledge, there has still not been any attempt to try to find and exploit such inconsistencies systematically.
In this chapter, we design a new DoS attack which exploits inconsistencies in the gas metering mechanism by taking a systematic approach to finding these.
We first replay and analyse several months of transactions to discover discrepancies in the gas cost.
We then use the data and insight from our analysis to design a genetic algorithm capable of generating low-throughput contracts.
We evaluate the contracts generated by our algorithm on all major Ethereum clients and find that they are all vulnerable to our attack.

\point{Contributions}
This chapter makes the following contributions:
\begin{enumerate}
    \item \textbf{Exploration of metering in EVM}: We explore the history of executing~\empirical{\Months} months worth of smart contracts on the Ethereum blockchain and identify several important edge cases that highlight inherent flaws in EVM metering; specifically, we identify~i) EVM instructions for which the gas fee is too low compared to their resources consumption; and~ii) cases of programs where the cache influences execution time by an order of magnitude.
    
    % \item \textbf{Analysis of Ethereum main net}:	 demonstrate that the gas usage is only \emph{marginally correlated} with the usage of resources such as CPU and memory and that the gas cost is dominated by the EVM storage.
	%\item 	Compare gas costs vs. cloud provider costs and costs on different CPUs.
	
    \item \textbf{Resource Exhaustion Attacks (REA) contract generation strategy}: We present a code generation strategy able to produce REA attacks of arbitrary length.
    Some of the complexity comes from the need to produce well-formed EVM programs which minimise the throughput.
    We propose an approach which combines empirical data and a genetic algorithm in order to generate contracts with low throughput.
    We explore the efficacy of our strategy as a function of the throughput in terms of gas per second of the generated programs.
	\item \textbf{Experimental evaluation}:
	We show that our REA can abuse imperfections in EVM's metering approach.
  Our genetic algorithm can generate programs with a throughput of~\empirical{1.25M} gas per second after a single generation.
  A minimum in our experiments is attained at generation~\empirical{243} with a block using around~\empirical{9.9M} gas and taking about~\empirical{93 seconds}.
  We show that our method generates contracts on average more than~\Slowdown times slower than typical contracts.
  Finally, we evaluate our low-throughput contracts on the major Ethereum clients and show that they are all vulnerable.
  Using commodity hardware, nodes would be unable to stay in sync when under attack.
\item \textbf{Disclosure and fixes}:
  We responsibly disclosed our attack to the Ethereum Foundation and were awarded a bug bounty reward of 5,000 USD.
  We discussed with the developers about the ongoing efforts as well as some potential fixes, and present some of the short-term and long-term fixes in this chapter.
\end{enumerate}

\point{Chapter Organisation} The rest of the chapter is organised as follows.
In \autoref{sec:background}, we provide background information about Ethereum and its metering scheme, as well as a few instances of how it has been exploited in the past.
In \autoref{sec:case-studies}, we present case studies based on measurements that we obtained by re-executing the Ethereum main chain.
In \autoref{sec:attack}, we present our Resource Exhaustion Attacks (REA) and the results we obtained. In \autoref{sec:design} we present short and long-term solutions to gas mispricing issues.
Finally, we present related work in \autoref{sec:related} and conclude in \autoref{sec:conclusion}.
