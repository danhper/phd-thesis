\section{Background}
\label{sec:background}
In this section, we provide insights into smart contract execution costs on the Ethereum main network. Then, we highlight some of the attacks which have been performed by abusing the gas mechanism.

\subsection{Gas Statistics}
Now that we presented the key points about metering in the EVM, we provide concrete numbers about different aspects of the gas price and transaction fees. In particular, we show the amount of transaction fees that a user would have to pay to have his transaction processed by the main Ethereum network.

To give a sense of the transaction fees, we show a variety of typical fees in Figure~\ref{tab:gas-fee}. The fees are divided depending on their gas price and gas consumption. The \textit{Low} gas price is close to the lowest price that can be paid to get the transaction accepted on the Ethereum blockchain. The \textit{High} gas price refers to the price that people would pay when they are extremely eager to get their transaction included, for example when competing with other users to have a transaction included first~\cite{gas-price-history}. The \textit{basic} transaction type refers to transactions consuming only the base amount of gas, without executing any instruction. This is typically the cost to send Ether to a contract or another party. The \textit{gas intensive} transaction type represents computationally expensive transactions, for example, verifying a zero-knowledge proof~\cite{aztec-protocol}. At the time of writing, the maximum amount of gas which can be used in a single block is~\empirical{10,000,000}, which means only~\empirical{20} such transactions could be included in a single block.

In Figure~\ref{tab:empirical-gas-fee}, we show the values of the gas price, gas used and transaction fee. In order to obtain results reflecting the current situation, we limit the analysis to recent blocks. We use all the transactions sent to contracts between September 30, 2019 and January 15, 2020. We find that the median gas price paid by a transaction's sender is around~\empirical{9.1} Gwei, which is around~\empirical{9} times more than the minimum possible fee. It is worth noting that when paying the minimum possible fee, the probability for the transaction to get included in the next block is relatively low and the transaction can therefore be delayed for several blocks: at the time of writing, about \empirical{40\%} of the last 200 blocks accepted a gas price of \empirical{1Gwei}~\cite{eth-gas-station}. This explains that users usually pay a higher fee to get their transaction included faster. The median for the gas consumed by contracts is around~\empirical{50,000} gas, indicating that most transactions perform relatively simple computations. Indeed, the basic fee being~21,000, a simple read followed by an allocation of storage would already result in~46,000 gas. Overall, the median fee paid per transactions is~\empirical{0.0008 ETH} which is around~\empirical{\ToUSD{0.0008} USD}.

% \subsection{Metering in Other Blockchains}
% WASM, EOS, https://github.com/ewasm/design(https://github.com/ewasm/design/blob/master/metering.md)

\subsection{Previously Known Attacks}
The Ethereum network has been victim of several Denial of Service (DoS) attacks due to instructions being under-priced. We present two considerable DoS attacks which were performed on the Ethereum network.

\point{\lstinline{EXTCODESIZE} attack}
In September 2016, a DoS attack was performed on the Ethereum network by flooding it with transactions containing a very large number of \lstinline{EXTCODESIZE} instructions~\cite{transaction-spam-attack}. \lstinline{EXTCODESIZE} is an instruction to retrieve the size in bytes of a given contract's code.

This attack happened because the \lstinline{EXTCODESIZE} instruction was vastly under-priced. At the time of the attack, a single execution of this instruction cost~20 gas, meaning that one could perform around~1,500 instructions with less than~\$0.01. Although by itself, this issue might seem benign, \lstinline{EXTCODESIZE} forces the client to search the contract on disk, resulting in IO heavy transactions. While replaying the Ethereum history on our hardware, the malicious transactions took around~20 to~80 seconds to execute, compared to a few milliseconds for the average transactions. We show the correlation between the clock time and the gas used by transactions during the period of the attack in Figure~\ref{fig:extcodesize-cpu}. Although this attack did not create any issue at the consensus layer, it reduced the rate of block creation by a factor of more than 2 times, with block creation time peaking to more than~30s~\cite{block-time-chart}.

The Ethereum protocol was updated in EIP~150, with all the software running Ethereum, to increase the price of the \lstinline{EXTCODESIZE} from~20 to~700 gas, making the aforementioned attack considerably more expensive to perform. Some performance improvements were also made at the implementation level, allowing clients to process IO-intensive instructions faster.

\point{\lstinline{SUICIDE} Attack}
Shortly after the \lstinline{EXTCODESIZE} attack, another DoS attack involving the \lstinline{SUICIDE} instruction was performed~\cite{suicide-attack}. The \lstinline{SUICIDE} instruction kills a contract and sends all its remaining Ether to a given address. If this particular address does not exist, a new address would be newly created to receive the funds. Furthermore, at the time of the attack, calling \lstinline{SUICIDE} did not cost any Ether. Given these two properties, an attacker could create and destroy a contract in the same transaction, creating a new contract each time at an extremely low fee. This quickly overused the memory of the nodes, and particularly affected the Go implementation~\cite{geth} which was less memory efficient~\cite{geth-memory-efficiency}.

A twofold fix was issued for this attack in EIP~150. First, and most importantly, \lstinline{SUICIDE} would be charged the regular amount of gas for contract creation when it tried to send Ether to a non-existing address. Subsequently, the price of the \lstinline{SUICIDE} instruction was increased from~0 to~5,000 gas. Again, these measures would make such an attack very expensive.
