\section{Related Work}
\label{sec:3:related}

There has been a great deal of attention focused on the correctness of smart contracts on blockchains, especially, the Ethereum blockchain. Some of the vulnerability types have to do with gas consumption, but not all. There has been relatively little attention given to the organisation of metering for blockchain systems. We will first highlight the work that focuses on metering at the smart contract level and then, we will present research focusing on metering at the virtual machine level~---~EVM in the case of Ethereum.

\subsection{Gas Usage and Metering}
Yang et al.~\cite{DBLP:journals/corr/abs-1905-00553} have recently empirically analysed the resource usage and gas usage of the EVM instructions.
They provide an in-depth analysis of the time taken for each instruction both on commodity and professional hardware. Although our work was performed independently, the results we present in \autoref{sec:3:case-studies} seem to concur mostly with their findings.

Other related themes have also been covered in the literature. One of the large themes is the optimisation of gas usage for smart contracts. Another one is estimating, preferably statically, the gas consumption of smart contracts.

\subsubsection*{Gas Usage Optimisation}
Gasper~\cite{Chen2017} is one of the first papers which has focused on finding gas-related anti-patterns for smart contracts. It identifies 7 gas-costly patterns, such as dead code or expensive operations in loops, which could potentially be costly to the contract developer in terms of gas. Gasper builds a control flow graph from the EVM bytecode and uses symbolic execution backed by an SMT solver to explore the different paths that might be taken.

MadMax~\cite{Grech2018} is a static analysis tool to find gas-focused vulnerabilities. Its main difference with Gasper from a functionality point of view is that MadMax tries to find patterns which could cause out-of-gas exceptions and potentially lock the contract funds, rather than gas-intensive patterns. For example, it can detect loops iterating on an unbounded number of elements, such as the number of users, and which would therefore always run out of gas after a certain number of users. MadMax decompiles EVM contracts and encodes properties about them into Datalog to check for different patterns. It is performant enough to analyse all the contracts of the Ethereum blockchain in only 10 hours.

\subsubsection*{Gas Estimation}
Marescotti et al.~\cite{10.1007/978-3-030-03427-6_33} propose two algorithms to compute the upper-bound gas consumption of smart contracts. It introduces a ``gas consumption path'' to encode the gas consumption of a program in its program path. It uses an SMT solver to find an environment resulting in a given path and computes its gas consumption. However, this work is not implemented with actual EVM code and is therefore not evaluated on real-world contracts.

Gastap~\cite{DBLP:journals/corr/abs-1811-10403} is a static analysis tool which allows to compute sound upper bounds for smart contracts.
This ensures that if the gas limit given to the contract is higher than the computed upper bound, the contract is assured to terminate without out-of-gas exception.
It transforms the EVM bytecode and models it in terms of equations representing the gas consumption of each instructions.
It then solves these equations using the equation solver PUBS~\cite{10.1007/978-3-540-69166-2_15}.
Gastap can compute a gas upper bound on almost all real-world contracts it is evaluated on.


\subsection{Virtual Machines and Metering}
Zheng et al.~\cite{8449244} propose a performance analysis of several blockchain systems which leverage smart contracts. Although the analysis goes beyond smart contracts metering, with metrics such as network-related performance, it includes an analysis of smart contracts metering at the virtual machine level. Notably, it shows that some instructions, such as \lstinline{DIV} and \lstinline{SDIV}, consume the same amount of gas while their consumption of CPU resources is vastly different.

Chen et al.~\cite{Chen2017Metering} propose an alternative gas cost mechanism for Ethereum. The gas cost mechanism is not meant to replace completely the current one, but rather to extend it in order to prevent DoS attacks caused by under-priced EVM instructions. The authors analyse the average number of execution of a single instruction in a contract, and model a gas cost mechanism to punish contracts which excessively execute a particular instruction. This gas mechanism allows normal contracts to almost not be affected by the price changes while mitigating spam attacks which have been seen on the Ethereum blockchain~\cite{transaction-spam-attack}.

\subsection{Follow-up work}
The work we presented in this chapter has inspired some further research in the area of smart contract metering.
We will present some of the work that has been done in relation to it.

In a study from Khan et al.~\cite{Khan2021GasCA}, 5,000 Solidity-based smart contract transactions were analysed to identify patterns that affect gas consumption.
The researchers performed statistical analyses, including correlation and regression, to investigate the relationship between Solidity parameters, opcodes, and gas usage.
They pinpointed factors that contribute to increases or decreases in gas consumption, with their regression analysis revealing that 87.8\% of the variability in gas consumption is attributed to the examined parameters.

The research conducted by Li et al.~\cite{10.1145/3460120.3485369} uncovers vulnerabilities in transaction handling across all known Ethereum clients, such as Geth. They exploit these flaws to design a series of low-cost denial-of-service attacks called DETER, which can disable a remote Ethereum node's txpool and disrupt critical downstream services, including mining, transaction propagation, and gas stations. DETER attacks are characterized by minimal or zero Ether cost and can potentially cause widespread disruption to the Ethereum network by targeting centralized services like mining pools and transaction relay services.
