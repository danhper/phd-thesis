\section{Related Work}
\label{sec:related}

There has been a great deal of attention focused on the correctness of smart contracts on blockchains, especially, the Ethereum blockchain. Some of the vulnerability types have to do with gas consumption, but not all. There has been relatively little attention given to the organisation of metering for blockchain systems. We will first present research focusing on smart contract issues, and then highlight the work that focuses on metering at the smart contract level. We will then present research focusing on metering at the virtual machine level~---~EVM in the case of Ethereum.

\subsection{Smart Contracts}
Major contracts vulnerabilities have been observed in recent years~\cite{Atzei2017} with sometimes multiple millions of dollars worth of Ether at stake~\cite{Securities2017,parity-wallet-freeze}.
One of the most famous exploit on the Ethereum blockchain was The DAO exploit~\cite{mehar2019understanding}, where an attacker used a re-entrancy vulnerability~\cite{Luu2016a,DBLP:conf/ndss/KalraGDS18} to drain funds out of The DAO smart contract. The attacker managed to drain more than 3.5 million of Ether, which would now be worth more than \ToUSD{3.5} million USD. Given the severity of the attack, the Ethereum community decided to hard-fork the blockchain, preventing the attacker to benefit from the Ether he had drained.

In order to prevent such exploits, many different tools have been developed over the years to detect vulnerabilities in smart contracts~\cite{harz2018towards}. One of the first tools which have been developed is Oyente~\cite{Luu2016a}. It uses symbolic execution to explore smart contracts execution pass and then uses an SMT solver~\cite{de2008z3} to check for several classes of vulnerabilities. Many other tools covering the same or other classes of vulnerabilities have also been developed~\cite{DBLP:conf/ndss/KalraGDS18,Brent2018,Tsankov:2018:SPS:3243734.3243780,Jiang:2018:CFS:3238147.3238177} and are usually based either on symbolic execution or static analysis methods such as data flow or control flow analysis. Some smart contract analysis tools have also focused more on analysing vulnerabilities related to gas~\cite{Grech2018,Chen2017,DBLP:journals/corr/abs-1811-10403}. We present some of these tools in the next subsection.

\subsection{Gas Usage and Metering}
Recent work by Yang et al.~\cite{DBLP:journals/corr/abs-1905-00553} have recently empirically analysed the resource usage and gas usage of the EVM instructions. They provide an in-depth analysis of the time taken for each instructions both on commodity and professional hardware. Although our work was performed independently, the results we present in Section~\ref{sec:case-studies} seem to concur mostly with their findings.

Other related themes have also been covered in the literature. One of the large theme is optimisation of gas usage for smart contracts. Another one is estimating, preferably statically, the gas consumption of smart contracts.

\subsubsection*{Gas Usage Optimisation}
Gasper~\cite{Chen2017} is one of the first paper which has focused on finding gas related anti-patterns for smart contracts. It identifies 7 gas-costly patterns, such as dead code or expensive operations in loops, which could potentially be costly to the contract developer in terms of gas. Gasper builds a control flow graph from the EVM bytecode and uses symbolic execution backed by an SMT solver to explore the different paths that might be taken.

MadMax~\cite{Grech2018} is a static analysis tool to find gas-focused vulnerabilities. Its main difference with Gasper from a functionality point of view is that MadMax tries to find patterns which could cause out-of-gas exceptions and potentially lock the contract funds, rather than gas-intensive patterns. For example, it is able to detect loops iterating on an unbounded number of elements, such as the numbers of users, and which would therefore always run out of gas after a certain number of users. MadMax decompiles EVM contracts and encodes properties about them into Datalog to check for different patterns. It is performant enough to analyse all the contracts of the Ethereum blockchain in only 10 hours.

\subsubsection*{Gas Estimation}
Marescotti et al.~\cite{10.1007/978-3-030-03427-6_33} propose two algorithms to compute upper-bound gas consumption of smart contracts. It introduces a ``gas consumption path'' to encode the gas consumption of a program in its program path. It uses an SMT solver to find an environment resulting in a given path and computes its gas consumption. However, this work is not implemented with actual EVM code and is therefore not evaluated on real-world contracts.

Gastap~\cite{DBLP:journals/corr/abs-1811-10403} is a static analysis tool which allows to compute sound upper bounds for smart contracts. This ensures that if the gas limit given to the contract is higher than the computed upper-bound, the contract is assured to terminate without out-of-gas exception. It transforms the EVM bytecode and models it in terms of equations representing the gas consumption of each instructions. It then solves these equations using the equation solver PUBS~\cite{10.1007/978-3-540-69166-2_15}. Gastap is able to compute gas upper bound on almost all real world contracts it is evaluated on.


\subsection{Virtual Machines and Metering}
Zheng et al.~\cite{8449244} propose a performance analysis of several blockchain systems which leverage smart contracts. Although the analysis goes beyond smart contracts metering, with metrics such as network related performance, it includes an analysis about smart contracts metering at the virtual machine level. Notably, it shows that some instructions, such as \lstinline{DIV} and \lstinline{SDIV}, consume the same amount of gas while their consumption of CPU resource is vastly different.

Chen et al.~\cite{Chen2017Metering} propose an alternative gas cost mechanism for Ethereum. The gas cost mechanism is not meant to replace completely the current one, but rather to extend it in order to prevent DoS attacks caused by under-priced EVM instructions. The authors analyse the average number of execution of a single instruction in a contract, and model a gas cost mechanism to punish contracts which excessively execute a particular instruction. This gas mechanism allows normal contracts to almost not be affected by the price changes while mitigating spam attacks which have been seen on the Ethereum blockchain~\cite{transaction-spam-attack}.
