The trustless nature of permissionless blockchains renders overcollateralization a key safety component relied upon by decentralized finance (DeFi) protocols.
Nonetheless, factors such as price volatility may undermine this mechanism. 
In order to protect protocols from suffering losses, undercollateralized positions can be \textit{liquidated}.
In this section, we present the first in-depth empirical analysis of liquidations on protocols for loanable funds (PLFs).
We examine Compound, one of the most widely used PLFs, for a period starting from its conception to September 2020.
We analyze participants' behavior and risk-appetite in particular, to elucidate recent developments in the dynamics of the protocol.
Furthermore, we assess how this has changed with a modification in Compound's incentive structure and show that variations of only~3\% in an asset's dollar price can result in over 10m USD becoming liquidable.
To further understand the implications of this, we investigate the efficiency of liquidators.
We find that liquidators' efficiency has improved significantly over time, with currently over~70\% of liquidable positions being immediately liquidated.
Lastly, we provide a discussion on how a false sense of security fostered by a misconception of the stability of non-custodial stablecoins, increases the overall liquidation risk faced by Compound participants.
