\subsection{Conclusion}
\label{sec:5b:conclusion}
In this section, we presented the first in-depth empirical analysis of liquidations on Compound, one of the largest PLFs in terms of total locked funds, from~\StartDate to~\EndDate.
We analysed agents' behaviour and in particular how much risk they are willing to take within the protocol.
Furthermore, we assessed how this has changed with the launch of the Compound governance token \coin{COMP}, where we found that agents take notably higher risks in anticipation of higher earnings.
This resulted in variations as little as 3\% in an asset's price being able to turn over 10 million USD worth of collateral liquidatable.
In order to better understand the potential consequences, we then measured the efficiency of liquidators, namely how quickly new liquidation opportunities are captured. Liquidators' efficiency was found to have improved significantly over time, reaching 70\% of instant liquidations.
Lastly, we demonstrated how overconfidence in the price stability of \coin{DAI}, increased the overall liquidation risk faced by Compound users.
Rather ironically, many users wishing to make the most of the new incentive scheme ended up causing higher volatility in \coin{DAI}---a dominant asset of the platform, resulting in the liquidation of their assets.
This is not Compound's misdoing, however, this highlights the to-date unknown dynamics of incentive structures across different DeFi protocols.
