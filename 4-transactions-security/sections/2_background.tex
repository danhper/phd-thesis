\section{Background}
\label{sec:background}

In this section, we briefly explain the fundamentals of permissionless blockchains and describe the structure of the three blockchain systems that we evaluate, highlighting their various design aspects. 

\subsection{Blockchain Basics}
In its essence, a blockchain is an append-only, decentralized database that is replicated across a number of computer nodes. 
Most blockchain systems record activities in the form of ``transactions''. 
A transaction typically contains information about its sender, its receiver, as well as the action taken, such as the transfer of an asset. 
Newly created transactions are broadcast across the network where they get validated by the participants. 
Valid transactions are grouped into data structures called \textit{blocks}, which are appended to the blockchain by referencing the most recent block.
Blocks are immutable, and state changes in the blockchain require new blocks to be produced.

Network latency and asynchrony inherent in the distributed nature of blockchains lead to a number of challenges. 
In particular, a blockchain must be able to reach consensus about the current state when the majority of participating nodes behave honestly. 
In order to resolve disagreement, a consensus protocol prescribing a set of rules is applied as part of the validation process. 

% To ensure consistency, there needs to be:

% \begin{enumerate}\itemsep=-1pt
% \item a set of rules to validate transactions \label{req1};
% \item a set of rules to validate blocks \label{req2};
% \item a mechanism to determine which chain of blocks represent. the current state \label{req3}
% \end{enumerate}
% %
% Requirement (\ref{req1}) is usually straightforward to implement, each participant simply verifies that the set of rules holds for each transaction.
% Most blockchains also have a simple solution for Requirement (\ref{req3}): they treat the longest existing chain of blocks as the valid one.
% The biggest challenge, and diversity of solutions arise when it comes to solving Requirement (\ref{req2}).
% Bitcoin provides the first practical solution to this problem, namely, Proof-of-Work~(PoW). Participating nodes who propose new blocks compete to solve a computationally expensive puzzle, and blocks are deemed valid if Requirement (i) they include a solution to this puzzle and (ii) all transactions they contain are valid.
The Proof-of-Work (PoW) consensus,
introduced by Bitcoin and currently also implemented by Ethereum, requires the participant to solve a computationally expensive puzzle to create a new block. Although PoW can maintain consistency well, it is by nature very time- and energy-consuming, which limits its throughput.
To preserve security while maintaining a sufficient degree of decentralization, scalability is often sacrificed~\cite{Xie2019}.
Indeed, the rate of block creation for both Bitcoin and Ethereum is relatively slow---on average 10 minutes and 14 seconds per block, respectively---and the only way to increase the throughput is to increase the size of a single block, allowing for more transactions per block.

Another issue related to blockchain systems is the need for all participants to replicate the data.
Since blockchains are append-only, participants need to ensure that their storage capacity keeps pace with the ever-increasing size of blockchain data. 
It is therefore crucial for blockchains to be designed in such a way that the storage used increases only moderately with time.


\subsection{Consensus Mechanisms}

In response to the scalability issues related to PoW, many blockchains have developed other mechanisms to ensure consensus, which allow higher rates of block creation.

\point{Delegated Proof-of-Stake~(DPoS) in EOSIO}
\label{sec:DPoS}
EOSIO uses the Delegated Proof-of-Stake~(DPoS) protocol which was first introduced in Bitshares~\cite{bitshares}.

Users of EOSIO, stake \coin{EOS} tokens to their favored block producers (BPs) and can choose to remove their stake at any time. 
The~21 BPs with the highest stake are allowed to produce blocks whereas the rest are put on standby. 
Blocks are produced in rounds of~126~($6 \times 21$). 
The order of block production is scheduled prior to each round and must be agreed upon by at least~15 block producers~\cite{EOS}.


\point{Liquid Proof-of-Stake (LPoS) in Tezos}
For its consensus mechanism, Tezos employs another variant of Delegated Proof-of-Stake: the Liquid Proof-of-Stake~(LPoS)~\cite{Tezos2018}.
Tezos' LPoS differs from EOSIO's DPoS in that with the former, the number of consensus participants---or ``delegates''---changes dynamically~\cite{Tezos2018, Goodman2014}.
This is because any node with a minimum amount of staked assets, arbitrarily defined to be~\empirical{8,000} \coin{XTZ} (about 16,000 USD at the time of writing \cite{CoinMarketCap2020}), is allowed to become a delegate, who then has the chance to be selected as either a ``baker'' or an ``endorser''.
Each block is produced (``baked'') by one randomly selected baker, and verified (``endorsed'') by 32 randomly selected endorsers~\cite{Tezos2018}. 
The endorsements are included in the following block.
% Ill-behaved bakers are punished with a deduction from their staked assets~\cite{Bernardo2019}.

% Blocks on Tezos are published in cycles. Each cycle consists of~4,096 blocks and takes approximately~3 days to complete~\cite{SunYin2018}.

% The ease with which participants are able to become bakers can enhance the security of the network. 
% With a high number of bakers, the chances of collusion are lower than in systems with a smaller number of block producers such as EOS.
% Tezos' LPoS is designed to be self-amending, meaning forks will never occur, which is an important aspect in terms of security. 
% In the case of a severe attack, such as those seen on Ethereum~\cite{Brent2018}, network participants could vote on-chain to modify the rules of the network to counter the given attack~\cite{Goodman2014}. 
% This makes Tezos a less attractive target for such attacks than blockchains without such self-amendment capabilities.

\point{XRP Ledger Consensus Protocol (XRP LCP) in XRPL}
XRPL is a distributed payment network created by \textit{Ripple Labs Inc.} in~2012 that uses the XRP ledger consensus protocol~\cite{Chase2018}.
Each user sets up its own unique node list of validators (UNL) that it will listen to during the consensus process. 
The validators determine which transactions are to be added to the ledger. 
Consensus is reached if at least~\empirical{90\%} of the validators in each ones' UNL overlap. If this condition is not met, the consensus is not assured to converge and forks can arise~\cite{Chase2018}.

% The network is decentralized as any user can become a validator and a user can freely select a set of validators for his UNL. 
% However, each pair of users must have an overlap of at least 40\% to meet the fault tolerance requirement~\cite{Chase2018}. 
% This allows the users to have diversity in the set of validators, which in turn improves the decentralization properties of the ledger.


% The size of the quorums determines the resistance of the protocol against malicious nodes. 
% The current quorum size is~80\% of the UNL size meaning that the proportion of misbehaving validators, the Ripple Consensus is exposed to, is limited to a maximum of just~20\%. 
% With this condition included, the consensus algorithm requires more than~90\% of UNL overlaps. 

% This vastly reduces the amount of diversity the network can have and hence, reduces the security provided by diversity.

% lost first few blocks

% \url{http://web.archive.org/web/20171211225452/https://forum.ripple.com/viewtopic.php?f=2&t=3613}

% \url{https://xrpl.org/intro-to-consensus.html}

% and a payment receiver can accept a currency to the amount specified by a trustline between the receiver and the sender.



% theft

% https://medium.com/@john.cantell/hi-renier-8f887aee027b

% price discovery or spam?!


% Table generated by Excel2LaTeX from sheet 'Sheet1'
\begin{table}[htbp]
	\centering
	\caption{Distribution of action types per blockchain.}
	\label{tab:transaction-types-distribution}%
	\setlength{\tabcolsep}{1.4pt}
	\scriptsize
	\begin{tabular}{@{}lp{0.7in}rrp{1.2in}rrp{0.9in}rr@{}}
		\toprule
		                 & \multicolumn{3}{c}{\textbf{EOSIO}} & \multicolumn{3}{c}{\textbf{Tezos}} & \multicolumn{3}{c}{\textbf{ XRPL }}                                                                                                                     \\
		Category         & Action name                        & \#                                 & \%                                  & Operation kind         & \#        & \%                 & Transaction type     & \#          & \%                 \\
		\cmidrule(r){1-1} \cmidrule(rl){2-4} \cmidrule(rl){5-7} \cmidrule(l){8-10}
		P2P transactions & \textbf{Transfer}                  & 8,479,573,653                      & \textbf{             96.2 }         & Transaction            & 1,941,230 & 21.4               & \textbf{Payment}     & 100,328,458 & \textbf{     36.9} \\
		                 &                                    &                                    &                                     &                        &           &                    & EscrowFinish         & 677         & 0.0                \\
		\cmidrule(r){1-1} \cmidrule(rl){2-4} \cmidrule(rl){5-7} \cmidrule(l){8-10}
		Account actions  & newaccount                         & 289,680                            & 0.0                                 & Reveal                 & 113,915   & 0.0                & TrustSet             & 3,339,620   & 1.2                \\
		                 & bidname                            & 244,248                            & 0.0                                 & Origination            & 3,159     & 1.3                & AccountSet           & 150,401     & 0.1                \\
		                 & deposit                            & 243,881                            & 0.0                                 & Activate               & 2,659     & 0.0                & SignerListSet        & 13,707      & 0.0                \\
		                 & linkauth                           & 148693                             & 0.0                                 &                        &           &                    & SetRegularKey        & 734         & 0.0                \\
		                 & updateauth                         & 136,926                            & 0.0                                 &                        &           &                    & DepositPreauth       & 3           & 0.0                \\
		\cmidrule(r){1-1} \cmidrule(rl){2-4} \cmidrule(rl){5-7} \cmidrule(l){8-10}
		Other actions    & delegatebw                         & 684,449                            & 0.0                                 & \textbf{Endorsement}   & 6,957,612 & \textbf{     76.6} & \textbf{OfferCreate} & 160,451,595 & \textbf{     59.1} \\
		                 & undelegatebw                       & 461,320                            & 0.0                                 & Delegation             & 56,336    & 0.6                & OfferCancel          & 7,259,908   & 2.7                \\
		                 & buyrambytes                        & 353,695                            & 0.0                                 & Reveal nonce           & 9,409     & 0.1                & EscrowCreate         & 1,393       & 0.0                \\
		                 & rentcpu                            & 187,878                            & 0.0                                 & Ballot                 & 514       & 0.0                & EscrowCancel         & 84          & 0.0                \\
		                 & voteproducer                       & 137,713                            & 0.0                                 & Proposals              & 90        & 0.0                & PaymentChannelClaim  & 172         & 0.0                \\
		                 & buyram                             & 89,971                             & 0.0                                 & Double baking evidence & 4         & 0.0                & PaymentChannelCreate & 33          & 0.0                \\
		                 & Others                             & 332,799,590                        & 3.8                                 &                        &           &                    & EnableAmendment      & 12          & 0.0                \\
		\midrule
		\midrule
		Total            &                                    & 8,815,351,697                      & 100.0                               &                        & 9,084,928 & 100.0              &                      & 271,546,797 & 100.0              \\
		\bottomrule
	\end{tabular}%
\end{table}

\subsection{Account and Transaction Types}
In this section we describe the types of transactions that exist on the three blockchains.

\subsubsection{EOSIO}
EOSIO differentiates between system and regular accounts.
The former are built-in accounts created when the blockchain was first instantiated, and are managed by currently active BPs, while the latter can be created by anyone.
System accounts are further divided into privileged and unprivileged accounts. 
Privileged accounts, including \texttt{eosio}, \texttt{eosio.msig}, and \texttt{eosio.wrap}, can bypass authorization checks when executing a transaction \cite{EOSIO2019, Kauffman2019} (see Section~\ref{sec:DPoS}).

EOSIO system contracts, defined in \texttt{eosio.contracts}~\cite{EOSIO2020}, are held by system accounts. 
One of the most commonly used system contracts is \texttt{eosio.token}, which is designed for creating and transferring user-defined tokens~\cite{EOSIO2019}.
Regular accounts can freely design and deploy smart contracts. 

Each smart contract on EOSIO has a set of actions. 
Actions included in non-system contracts are entirely user-defined, and users have a high degree of flexibility in terms of structuring and naming the actions. 
This makes the analysis of actions challenging, as it requires understanding their true functionality on a case-by-case basis. 
While many actions have a candid name that gives away their functionality (e.g. \texttt{payout} from contract \texttt{betdicegroup}), some are less expressive (e.g. \texttt{m} from user \texttt{pptqipaelyog}).

In \autoref{tab:transaction-types-distribution}, we show different types of existing actions. 
Since actions from non-system contracts have arbitrary designs, we only examine actions that belong to system accounts for the moment, as these are already known and easier to classify. We make one exception to this and include the actions of \emph{token} contracts, as they have a standardized interface~\cite{eosio-tokens}.
Overall, we can see that token transfers account alone for more than~\empirical{96\%} of the transactions. The rest of the transactions are mostly user-defined and appear under ``Others'' in the table, while actions defined in system contracts only account for a very small percentage of the entire traffic volume.

% In particular, we choose all the actions related to resource allocations, such as \texttt{buyram} or \texttt{delegatebw}. 
% We also include the \texttt{transfer} action, which is the action used in EOS to transfer assets. 

%We will cover the distribution of these in more detail in the following section.

\subsubsection{Tezos}
Tezos has two types of accounts: implicit and originated.
Implicit accounts are similar to the type of accounts found in Ethereum, generated from a public-private key pair~\cite{Wood2014}.
These accounts can produce---or ``bake''---blocks and receive stakes, but cannot be used as smart contracts.
Bakers' accounts must be implicit, to be able to produce blocks.
Originated accounts are created and managed by implicit accounts, but do not have their own private key~\cite{NomadicLabs2018a}. 
They can function as smart contracts, and can delegate voting rights to bakers' implicit accounts~\cite{Labs2018}.

``Transactions'' on Tezos are termed  ``operations.''
Operations can be roughly classified into three types: consensus related, governance related and manager operations~\cite{AmitPanghal2019}.
Consensus-related operations, as the name indicates, ensure that all participating nodes agree on one specific version of data to be recorded on the blockchain. 
Governance-related operations are used to propose and select a new set of rules for the blockchain. 
However, these events are very rare and only involve bakers, which is why these operations only represent a low percentage of the total number of transactions. 
Operations mainly consist of delegations and peer-to-peer payment transactions. 
As shown in \autoref{tab:transaction-types-distribution}, \texttt{endorsement} operations account for a vast majority,~\empirical{76\%}, of total operations.
% which are used to determine the next block header selected ,
Endorsements are performed by bakers, and a block needs a minimum of 32 endorsements for it to be accepted~\cite{NomadicLabs2018b}.

% Transactions, which represent almost all the rest of the volume, are used to transfer assets from one account to another. 


\subsubsection{XRPL}
XRPL also uses an account-based system to keep track of asset holdings. 
Accounts are identified by addresses derived from a public and private key pair. 
There are a handful of ``special addresses'' that are not derived from a key pair. 
Those addresses either serve special purposes (e.g. acting as the \coin{XRP} issuer) or exist purely for legacy reasons. 
Since a secret key is required to sign transactions, funds sent to any of these special addresses cannot be transferred out and are hence permanently lost~\cite{XRPLedger2019}.

XRPL has a large number of predefined transaction types. 
We show part of them in \autoref{tab:transaction-types-distribution}. 
The most common transaction types are \texttt{OfferCreate}, which is used to create a new order in a decentralized exchange (DEX) on the ledger, and \texttt{Payment}, which is used to transfer assets. 
There are also other types of transactions such as \texttt{OfferCancel} used to cancel a created order or \texttt{TrustSet} which is used to establish a ``trustline''~\cite{xrp_ledger_overview} with another account.

\subsection{Expected Use Cases}
\label{sec:usecase}
In this section, we describe the primary intended use cases of the three blockchains and provide a rationale for the way they are being used, to better understand the dynamics of actual transactions evaluated in Section~\ref{sec:data-analysis}.

\point{EOSIO} EOSIO was designed with the goal of high throughput and has a particularity compared to many other blockchains: there are no direct transaction fees. 
Resources such as CPU, RAM and bandwidth are rented beforehand, and there is no fixed or variable fee per transaction~\cite{EOS}.
This makes it a very attractive platform for building decentralized applications with a potentially high number of micro-payments. 
Many games, especially those with a gambling nature, have been developed using EOSIO as a payment platform.
EOSIO is also used for decentralized exchanges, where the absence of fees and the high throughput allow placing orders on-chain, unlike many decentralized exchanges on other platforms where only the settlement is performed on-chain~\cite{warren20170x}.

\point{Tezos}
Tezos was one of the first blockchains to adopt on-chain governance. 
This means that participants can vote to dynamically amend the rules of the consensus. 
A major advantage of this approach is that the blockchain can keep running without the need of hard forks, as often observed for other blockchains~\cite{byzantium-fork, dao-fork}. 
Another characteristic of Tezos is the use of a strongly typed programming language with well-defined semantics~\cite{NomadicLabs2018} for its smart contracts, which makes it easier to provide these for correctness. 
These properties make Tezos a very attractive blockchain for financial applications, such as the tokenization of assets~\cite{BTGPactual2019}.

\point{XRPL} 
Similar to EOSIO, XRPL supports the issuance, circulation, and exchange of customized tokens. 
However, in contrast to EOSIO, XRPL uses the IOU (``I owe you'') mechanism for payments. 
Specifically, any account on XRPL can issue an IOU with an arbitrary ticker~---~be it \coin{USD} or \coin{BTC}.
Thus, if Alice pays Bob~10 \coin{BTC} on XRPL, she is effectively sending an IOU of~10 \coin{BTC}, which literally means ``I (Alice) owe you (Bob)~10 \coin{BTC}''. 
Whether the \coin{BTC} represents the market value of Bitcoin depends on Alice's ability to redeem her ``debt''~\cite{XRPLedger2020}. 
This feature contributes to the high throughput on XRPL, as the speed to transfer a specific currency is no more constrained by its original blockchain-related limitations: For example, the transfer of \coin{BTC} on XRPL is not limited by the block production interval of the actual Bitcoin blockchain (typically~10 minutes to an hour to fully commit a block), and the transfer of \coin{USD} is not limited to the speed of the automated clearing house~(ACH) (around two days~\cite{Love2013}).

% XRP has been built from the ground-up to facilitate cross-border cross-currency transfers~\cite{xrp_ledger_overview}. 
% As seen earlier, \texttt{Payment} is one of the most commonly used transaction types, and given the distributed nature of XRP, it allows for cross-border transfers. 
% One of the features which enable cross-currency transfers is the on-chain decentralized exchange. 
% Users have the ability to place either limit or market orders directly on the ledger. 
% Assuming that the liquidity of the asset to exchange is high enough, this allows for almost instantaneous cross-currency transfer. 
% Obviously, the on-chain settlements only represent a promise of payment and the conversion to actual fiat currency must be done through a ``gateway'' holding funds outside of the ledger.


% \point{Summary}
% Given the expected use cases of these different blockchains, we expect to see some variation in the transaction patterns and how the transactional throughput is used in general. 
% In particular, the absence of fees for EOS transactions suggests that the number of transactions is likely to be significantly higher than for other blockchains. 
% In the upcoming section, we will use the empirical data we have collected to investigate these different expected trends.
