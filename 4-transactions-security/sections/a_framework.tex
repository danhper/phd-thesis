\section{Measurement Framework}
We implement an extensible and reusable measurement framework to facilitate future transactions analysis related research.
Our framework currently supports the three blockchains analyzed in this work, Tezos, EOSIO and XRPL but can easily be extended to support other blockchains.
The core of the software is implemented in Go and is designed to work well on a single machine with many cores.
The framework frontend is provided as a cross-platform static binary command line tool.

While the framework is responsible for the heavy lifting and processing the gigabytes of data, we also provide a companion tool implemented in Python to generate plots and tables from the data generated by the framework.
\point{Data fetching}
The framework currently allows to fetch data either using RPC over HTTP or websockets.
Tezos and EOSIO both use the HTTP adapter to retrieve data while XRPL uses the websocket interface.
The data is retrieved from publicly available archive nodes but the framework can be configured to use other nodes if necessary.
The retrieved data is stored in a gzipped JSON Lines format where each line corresponds to a block. Blocks are stored in chunk of $n$ blocks per file --- where $n$ can be configured --- making parallel processing straightforward. It took less than two days to fetch all the data presented in~\autoref{tab:data-summary}.

\begin{figure}[h!]
\begin{lstlisting}[language=json]
{
  "Pattern": "/data/eos_blocks-*.jsonl.gz",
  "StartBlock": 82152667,
  "EndBlock": 118286375,
  "Processors": [{
      "Name": "TransactionsCount",
      "Type": "count-transactions"
    }, {
      "Name": "GroupedActionsOverTime",
      "Type": "group-actions-over-time",
      "Params": {
        "By": "receiver",
        "Duration": "6h"
      }
    }, {
      "Name": "ActionsByName",
      "Type": "group-actions",
      "Params": {
        "By": "name"
      }
    }
}
\end{lstlisting}
  \caption{Configuration file for our measurement framework}
  \label{lis:framework-config}
\end{figure}

\point{Data processing}
The framework provides several processors which can mainly be used to aggregate the data either over time, or over certain properties such as the sender of a transaction.
The framework is configured using a single JSON file, containing the configuration for the data to be processed as well as the specification of what type of statistics should be collected from the dataset. We show a sample configuration file in~\autoref{lis:framework-config}.
This configuration computes three statistics from block~82,152,667 to block~118,286,375, using the data contained in all the files matching~\lstinline{/data/eos_blocks-*.jsonl.gz}.
The framework will compute the total number of transactions, the number of actions grouped using their receiver over a period of 6 hours, and finally the total number of actions grouped by their name.
All the statistics described above can be used for all the blockchains but the framework also supports blockchain-specific statistics where needed.
New statistics can easily be added to the framework by implementing a common interface.

Our framework was able to analyze the data and output all the statistics required for this paper in less than 4 hours using a powerful 48 core machine.

\begin{figure}[ht]
\begin{lstlisting}[language=go]
type Blockchain interface {
    FetchData(filepath string,
              start, end uint64) error
    ParseBlock(rawLine []byte) (Block, error)
    EmptyBlock() Block
}

type Block interface {
    Number() uint64
    TransactionsCount() int
    Time() time.Time
    ListActions() []Action
}

type Action interface {
    Sender() string
    Receiver() string
    Name() string
}
\end{lstlisting}
  \caption{Main interfaces of our measurement framework}
  \label{lis:framework-interfaces}
\end{figure}

\point{Extending to other blockchains}
The framework has been made as generic as possible to allow integrating other blockchains to perform similar kind of analysis.
In particular, the framework contains three main interfaces shown in~\autoref{lis:framework-interfaces}.
The \lstinline{FetchData} method can be implemented by reusing the HTTP or websocket adapters provided by the framework while the \lstinline{Block} and \lstinline{Action} interfaces typically involves defining the schema of the block or action of the blockchain implemented.
In our implementation, adding a blockchain takes on average 105 new lines of Go code not including tests.
