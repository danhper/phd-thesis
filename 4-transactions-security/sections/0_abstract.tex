Scalability has been a bottleneck for major blockchains such as Bitcoin and Ethereum.
As we have seen in the previous chapter, the execution layer can be a major bottleneck for scalability and even potentially lead to DoS attacks.
Some newer blockchains have improved scalability and allowed for much higher transactional throughput.
However, there has been little effort to understand how their transactional throughput is being used.
In this chapter, we examine recent network traffic of three major high-scalability blockchains---EOSIO, Tezos and XRP Ledger (XRPL)---over a period of seven months.
Our analysis reveals that only a small fraction of the transactions are used for value transfer purposes. In particular, \empirical{96\%} of the transactions on EOSIO were triggered by the airdrop of a currently valueless token; on Tezos,~\empirical{76\%} of throughput was used for maintaining consensus; and over \empirical{94\%} of transactions on XRPL carried no economic value. We also identify a persisting airdrop on EOSIO as a DoS attack and detect a two-month-long spam attack on XRPL.
The chapter explores the different designs of the three blockchains and sheds light on how they could shape user behaviour.
