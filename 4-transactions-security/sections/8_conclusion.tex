\section{Conclusions}
\label{sec:conclusion}

We investigate transaction patterns and value transfers on the three major high-throughput blockchains: EOSIO, Tezos, and XRPL. 
Using direct connections with the respective blockchains, we fetch transaction data between~\startdate and \finishdate.
With EOSIO and XRPL, the majority of the transactions exhibit characteristics resembling DoS attacks: on EOSIO,~\empirical{95\%} of the transactions were triggered by the airdrop of a yet valueless token; on XRPL, over~\empirical{94\%}---consistently in different observation periods---of the transactions carry no economic value. 
For Tezos, since transactions per block are largely outnumbered by mandatory endorsements, most of the throughput,~\empirical{76\%} to be exact, is occupied for maintaining consensus.

Furthermore, through several case studies, we present prominent cases of how transactional throughput was used on different blockchains.
Specifically, we show two cases of spam on EOSIO, on-chain governance related transactions on Tezos, as well as payments and exchange offers with zero-value tokens on XRPL.

The bottom line is: the three blockchains studied in this paper demonstrate the capacity to support high levels of throughput; however, the massive potential of those blockchains has thus far not been fully realized for their intended purposes.

% The potential of the
% Endurance capacity


%Moreover, this paper focuses on the notion of \emph{useful throughput}, that is to say, throughput of transactions that can be deemed to be performing useful work. 
