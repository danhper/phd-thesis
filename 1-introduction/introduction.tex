\chapter{Introduction}

\section{Motivation and Objectives}
During the past several years, blockchain systems have gained a lot of traction and adoption, with during peak period, over 100 billion dollars worth of assets sitting in blockchain systems.
Given he permisionless nature blockchain systems and their large scope in terms of software -- e.g. distributed consensus, untrusted program execution -- there are numerous attack vectors that need to be studied, understood and protected against for blockchain systems to be able to deliver their promises of a safer financial system.

The high-level goal of this thesis is to improve our understanding of the different that allows blockchain systems to operate decentralized financial applications, and ultimately make blockchain systems, and the applications built on top of them more secure.



\section{Contributions}

Contributions here.


\section{Statement of Originality}

Statement here.


\section{Publications}

In this thesis, we include the following papers:

\begin{itemize}
\item Broken Metre: Attacking Resource Metering in EVM, NDSS'20
\item Smart Contract Vulnerabilities: Vulnerable Does Not Imply Exploited, USENIX Security'21
\item Revisiting Transactional Statistics of High-scalability Blockchains, IMC'21
\item Liquidations: DeFi on a Knife-edge, FC'21
\item Dissimilar Redundancy in DeFi, MARBLE'22
\end{itemize}
