\section{Technical and Economic Security}
\subsection{Technical Security}
\label{sec:5:technical-security}

We define a security risk to be \emph{technical} if an agent can atomically exploit a protocol.
In a technical exploit, an attacker effectively finds a sequence of contract calls, whether in a single transaction or a bundle of transactions, that leads to a profit through a violation of a protocol's intended properties (as visualized in Fig.~\ref{fig:technical-security-illustration}).
Such exploits can be performed risk-free (and often in a sense `instantaneously') because the outcomes for the attacker are binary: either the attack is successful and the attacker profits or the transaction reverts (effectively the attack doesn't happen) and the attacker only loses some gas fees.

\begin{figure}[htp]
  \centering
      \begin{tikzpicture}[auto,
                     > = Stealth, 
         node distance = 10mm,
            box/.style = {draw=gray, very thick,
                          minimum height=11mm, text width=22mm, 
                          align=center},
     every edge/.style = {draw, very thick},
every edge quotes/.style = {font=\footnotesize, align=center, inner sep=1pt}
                          ]
% from bottom to top
  \node (n11) [devil,evil,minimum size=1cm] {};
  \node (n12) [box, right=of n11] { Contract 1};
  \node (n13)[box, right=of n12] {Contract 2};
  \node (n21) [box, below=of n12] {Contract N};
  \node (n22) [box, below=of n13] {Contract 3};
%Lines
\draw
      (n11) edge [<->, "\$\$"]        (n12)
      (n12) edge [->]        (n13)
      (n13) edge [->]        (n22)
      (n22) edge [dotted]    (n21)
      (n21) edge [->]        (n12);      
      \end{tikzpicture}
      \caption{Diagram of a technical exploit.}
      \label{fig:technical-security-illustration}
\end{figure}

In current blockchain implementations, this coincides with (1) manipulating an on-chain system within a single transaction, which is risk-free for anyone, and (2) manipulating transactions within the same block, which is risk-free for the miner generating that block or for an attacker who creates a bundle of transactions that are required to execute atomically in the order given.
By exploiting technical structure, the underlying blockchain system allows no opportunity for markets or other agents to react in the course of such exploits (when such reaction is possible, we enter the domain of \emph{economic} security problems in the next section).
When there is competition to perform these exploits, they will factor into the game theory of blockchain mining (e.g., \cite{biais2019blockchain}) as part of MEV extraction (as discussed in \cite{daian2019flash}); however, attempting these exploits will be risk-free (minus potential gas fees) for any attacker.
We identify three categories of attacks that fall within technical security risks of protocols: attacks exploiting smart contract vulnerabilities, attacks relying on the execution order of transactions in a block, as well as attacks which are executed within a single transaction.
These security risks are often addressable with program analysis and formal models to specify protocols, although these problems can quickly become complex to formulate and computationally hard.

\begin{tcolorbox}[boxsep=1pt,left=2pt,right=2pt,top=2pt,bottom=2pt, title=Technical Security]
\emph{A protocol is technically secure if it is not possible for an attacker to atomically exploit the protocol at the expense of value held by the protocol or its users. Due to atomicity, these attacks can generate risk-free profit. A common property of technical exploits is that they occur within a single transaction or a bundle of transactions in a block.}
\end{tcolorbox}


\subsection{Economic security}
\label{sec:5:economic-security}

We define a security risk to be \emph{economic} if an attacker can perform a strictly non-atomic exploit to realize a profit at the expense of value held by the protocol or its users.
In an economic exploit, an attacker performs multiple actions at different places in the transaction sequence but doesn't control what happens between their actions in the sequence, which means that there is no guarantee that the final action is profitable (as visualized in Fig~\ref{fig:economic-security-illustration2} and in comparison to the technical exploit in Fig~\ref{fig:technical-security-illustration}).
Economic security is effectively about an exploiting agent who tries to manipulate a market or incentive structure over some time period (even if short, it is not instantaneous). Compared to technical exploits, since economic exploits are non-atomic, they come with upfront tangible costs, a probability of attack failure and risk related to mis-estimating the market response. Thus they are not risk-free and commonly involve manipulations over many transactions or blocks.

\begin{figure}[htp]
    \centering
        \begin{tikzpicture}[auto,
                       > = Stealth, 
           node distance = 10mm,
              box/.style = {draw=gray, very thick,
                            minimum height=6mm, text width=15mm, 
                            align=center},
       every edge/.style = {draw, very thick},
  every edge quotes/.style = {font=\footnotesize, align=center, inner sep=1pt}
                            ]
  % from bottom to top
    \node (n11) [] {t = 0};
    \node (n12) [devil,evil,minimum size=6mm, right=of n11] {};
    \node (n13) [box, right=of n12] {Contract 1};
    \node (n14) [box, right=of n13] {Contract N};
    \node (n21) [below=of n11] {...};
    \node (n22) [below=of n12] {};
    \node (n23) [below=of n13] {Market conditions change};
    \node (n24) [below=of n14] {};
    \node (n31) [below=of n21] {t = i};
    \node (n32) [devil,evil,minimum size=6mm, below=of n22] {};
    \node (n33) [box, below=of n23] {Contract 1};
    \node (n34) [box, below=of n24] {Contract 2};
    \node (n41) [below=of n31] {};
    \node (n42) [below=of n32] {};
    \node (n43) [box, below=of n33] {Contract N};
    \node (n44) [box, below=of n34] {Contract 3};
  %Lines
  \draw
        (n12) edge [->]     (n13)
        (n13) edge [dashed, ->] (n14)
        (n32) edge [<->, "\$\$ ?"]                   (n33)
        (n33) edge [->]                   (n34)
        (n44) edge [->]               (n34)
        (n34) edge [->]                   (n44)
        (n44) edge [dotted]                   (n43)
        (n43) edge [->]       (n33);      
        \end{tikzpicture}
        \caption{Diagram of an economic exploit.}
        \label{fig:economic-security-illustration2}
  \end{figure}
  

In addition to comparing the structures in Figures \ref{fig:technical-security-illustration} and \ref{fig:economic-security-illustration2}, we provide a simple example to help illustrate the distinction between technical and economic security. Consider a protocol that uses an instantaneous AMM price as an oracle. An attacker can perform a (atomic) sandwich attack to steal assets, which amounts to a technical exploit. If instead the protocol used a time-weighted average AMM price as an oracle, then the attacker could manipulate this price over time (non-atomically) and may still be able to steal assets, which would amount to an economic exploit.

Economic risks are inherently a problem of economic design and cannot be solved by technical means alone. To illustrate, while these attacks could be performed atomically (and risk-free) in a very poorly constructed system that allowed it, they are not solved, for example, just by adding a time delay that ensures they are not executed in the same block. Even if all technical issues are sorted, we are often left with remaining economic problems about how markets or other incentive structures could be manipulated over time to exploit protocols.
From a practical perspective, progress on these economic problems inherently requires economic models of these market equilibria and the design of better protocol incentive structures. These models differ considerably from traditional security models and are sparsely studied. As a result, defensive measures for economic security risks are also not as well established.

In this way also, technical security must be a first bar: if a protocol is not technically secure, then it will break in the presence of rational agents. Economic security only makes sense if technical security is achieved. For instance, if a protocol's funds can be exploited because it is not technically secure, then in an economic sense no rational agents should participate.


\begin{tcolorbox}[boxsep=1pt,left=2pt,right=2pt,top=2pt,bottom=2pt, title=Economic Security]
\emph{A protocol is economically secure if it is economically infeasible (e.g., unprofitable) for an attacker to perform exploits that are strictly non-atomic at the expense of value held by the protocol or its users. As economic exploits are non-atomic (or else they are better described as technical), they are not risk-free.}
\end{tcolorbox}
