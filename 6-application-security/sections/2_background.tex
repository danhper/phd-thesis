\section{Background}
\label{sec:background}
In this section we introduce preliminary concepts about blockchains and smart contracts necessary to the understanding of the rest of the paper.

\subsection{Blockchain}

% \point{Blockchain}
A blockchain, such as Bitcoin~\cite{Nakamoto} or Ethereum~\cite{wood2014ethereum}, is in essence a decentralized append-only database.
Data is added to the blockchain in the form of transactions that are grouped in blocks.
Some rules are enforced by the protocols on both transactions and blocks to ensure its correct working.
Blockchains need to be able to maintain consensus of which blocks are included.
Both Bitcoin and Ethereum use the Proof-of-Work consensus that requires block producers, often called \emph{miners}, to solve a computationally expensive puzzle to produce a new block \cite{Perez2020c}.
An important point to note is that miners are allowed to choose which transactions to include in a block and in which order to include them.
This can potentially allow miners to profit from having a transaction included before another one.
This is commonly referred to as \emph{miner-extractable value}~\cite{daian2020flash}.

\subsection{Smart Contracts}
\point{Ethereum smart contracts}
On Ethereum, smart contracts are programs written in a Turing-complete language, typically Solidity~\cite{docs:solidity}, that define a set of rules that may be invoked by any network participant. 
These programs rely on the Ethereum Virtual Machine (EVM), a low-level stack machine which executes the compiled EVM bytecode of a smart contract~\cite{wood2014ethereum}.
Each instruction has a fee measured in so-called \textit{gas}, and the total gas cost of a transaction is a fixed base fee plus the sum of all instructions' gas~\cite{albert2020gasol,Perez2020f}.
The sender of a transaction must then set a gas price, the amount of \coin{ETH} he is willing to pay per unit of gas consumed for executing the transaction.
The transaction fee is thus given by the gas price multiplied with the gas cost~\cite{werner2020step,pierro2019influence}.
Within a transaction, smart contracts can store data in logs, which are metadata specially indexed as part of the transaction.
This metadata, commonly referred to as \textit{events}, is typically used to allow users to monitor the activity of a contract externally.

\point{Oracles}
One of the major challenges smart contracts face concerns access to off-chain information, i.e. data that does not natively exist on-chain.
Oracles are data feeds into smart contracts and provide a mechanism for accessing off-chain information through some third party.
In DeFi, oracles are commonly used for price feed data to determine the real-time price of assets.
For instance, via the Compound Open Price Feed~\cite{web:compoundfinance_prices}, vetted third party reporters sign off on price data using a known public key, where the resulting feed can be relied upon by smart contracts.

\point{Stablecoins}
An alternative to volatile cryptoassets is given by stablecoins, which are priced against a peg and can be either custodial or non-custodial.
For custodial stablecoins (e.g. \coin{USDC}~\cite{web:usdc}), tokens represent a claim of some off-chain reserve asset, such as fiat currency, which has been entrusted to a custodian.
Non-custodial stablecoins (e.g. \coin{DAI}~\cite{whitepaper:maker}) seek to establish price stability via economic mechanisms specified by smart contracts.
For a thorough discussion on stablecoin design, we direct the reader to \cite{Klages-Mundt2020}.
